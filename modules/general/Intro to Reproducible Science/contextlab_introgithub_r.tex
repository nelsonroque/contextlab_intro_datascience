% Options for packages loaded elsewhere
\PassOptionsToPackage{unicode}{hyperref}
\PassOptionsToPackage{hyphens}{url}
%
\documentclass[
  ignorenonframetext,
]{beamer}
\usepackage{pgfpages}
\setbeamertemplate{caption}[numbered]
\setbeamertemplate{caption label separator}{: }
\setbeamercolor{caption name}{fg=normal text.fg}
\beamertemplatenavigationsymbolsempty
% Prevent slide breaks in the middle of a paragraph
\widowpenalties 1 10000
\raggedbottom
\setbeamertemplate{part page}{
  \centering
  \begin{beamercolorbox}[sep=16pt,center]{part title}
    \usebeamerfont{part title}\insertpart\par
  \end{beamercolorbox}
}
\setbeamertemplate{section page}{
  \centering
  \begin{beamercolorbox}[sep=12pt,center]{part title}
    \usebeamerfont{section title}\insertsection\par
  \end{beamercolorbox}
}
\setbeamertemplate{subsection page}{
  \centering
  \begin{beamercolorbox}[sep=8pt,center]{part title}
    \usebeamerfont{subsection title}\insertsubsection\par
  \end{beamercolorbox}
}
\AtBeginPart{
  \frame{\partpage}
}
\AtBeginSection{
  \ifbibliography
  \else
    \frame{\sectionpage}
  \fi
}
\AtBeginSubsection{
  \frame{\subsectionpage}
}
\usepackage{amsmath,amssymb}
\usepackage{lmodern}
\usepackage{iftex}
\ifPDFTeX
  \usepackage[T1]{fontenc}
  \usepackage[utf8]{inputenc}
  \usepackage{textcomp} % provide euro and other symbols
\else % if luatex or xetex
  \usepackage{unicode-math}
  \defaultfontfeatures{Scale=MatchLowercase}
  \defaultfontfeatures[\rmfamily]{Ligatures=TeX,Scale=1}
\fi
% Use upquote if available, for straight quotes in verbatim environments
\IfFileExists{upquote.sty}{\usepackage{upquote}}{}
\IfFileExists{microtype.sty}{% use microtype if available
  \usepackage[]{microtype}
  \UseMicrotypeSet[protrusion]{basicmath} % disable protrusion for tt fonts
}{}
\makeatletter
\@ifundefined{KOMAClassName}{% if non-KOMA class
  \IfFileExists{parskip.sty}{%
    \usepackage{parskip}
  }{% else
    \setlength{\parindent}{0pt}
    \setlength{\parskip}{6pt plus 2pt minus 1pt}}
}{% if KOMA class
  \KOMAoptions{parskip=half}}
\makeatother
\usepackage{xcolor}
\IfFileExists{xurl.sty}{\usepackage{xurl}}{} % add URL line breaks if available
\IfFileExists{bookmark.sty}{\usepackage{bookmark}}{\usepackage{hyperref}}
\hypersetup{
  pdftitle={Intro to Reproducible Science: Github, R},
  pdfauthor={Nelson Roque, PhD},
  hidelinks,
  pdfcreator={LaTeX via pandoc}}
\urlstyle{same} % disable monospaced font for URLs
\newif\ifbibliography
\setlength{\emergencystretch}{3em} % prevent overfull lines
\providecommand{\tightlist}{%
  \setlength{\itemsep}{0pt}\setlength{\parskip}{0pt}}
\setcounter{secnumdepth}{-\maxdimen} % remove section numbering
\ifLuaTeX
  \usepackage{selnolig}  % disable illegal ligatures
\fi

\title{Intro to Reproducible Science: Github, R}
\author{Nelson Roque, PhD}
\date{2022-04-15}

\begin{document}
\frame{\titlepage}

\begin{frame}{Intro to Reproducible Science: Overview}
\protect\hypertarget{intro-to-reproducible-science-overview}{}
\begin{itemize}
\tightlist
\item
  What is reproducible science? Why should you care?
\item
  What is R?
\item
  What is RMarkdown?
\item
  What is Endnote?
\item
  What is Github?
\item
  Using Github to pull/push code
\item
  Hands On: Intro to R syntax + packages
\item
  Workshop Plug
\item
  Resources
\end{itemize}
\end{frame}

\begin{frame}{What is reproducible science?}
\protect\hypertarget{what-is-reproducible-science}{}
Making entire scientific process transparent (when and as allowable by
law, grants).

\begin{itemize}
\tightlist
\item
  Sharing experiment and analysis code
\item
  Detailed methods sections
\item
  Sharing stimulus sets
\item
  Being able to walk through a data analysis start (load raw data) to
  finish (manuscript analyses) in code
\item
  Following FAIR Principles
\item
  Pre-registering hypotheses and analysis plans (e.g., on OSF)
\end{itemize}

\begin{block}{Why should you care?}
\protect\hypertarget{why-should-you-care}{}
Ioannidis, John P A. 2005. ``Why Most Published Research Findings Are
False.'' PLoS Medicine 2 (8): e124.
\url{doi:10.1371/journal.pmed.0020124}.

Open Science Collaboration. 2015. ``Estimating the Reproducibility of
Psychological Science.'' Science 349 (6251): aac4716--aac4716.
\url{doi:10.1126/science.aac4716}.
\end{block}
\end{frame}

\begin{frame}{FAIR Principles}
\protect\hypertarget{fair-principles}{}
\begin{block}{F: Findable}
\protect\hypertarget{f-findable}{}
``The first step in (re)using data is to find them. Metadata and data
should be easy to find for both humans and computers. Machine-readable
metadata are essential for automatic discovery of datasets and
services''
\end{block}

\begin{block}{A: Accessible}
\protect\hypertarget{a-accessible}{}
``Once the user finds the required data, she/he/they need to know how
they can be accessed, possibly including authentication and
authorisation.''
\end{block}

\begin{block}{I: Interoperable}
\protect\hypertarget{i-interoperable}{}
``The data usually need to be integrated with other data. In addition,
the data need to interoperate with applications or workflows for
analysis, storage, and processing.''
\end{block}

\begin{block}{R: Reusable}
\protect\hypertarget{r-reusable}{}
``The ultimate goal of FAIR is to optimise the reuse of data. To achieve
this, metadata and data should be well-described so that they can be
replicated and/or combined in different settings.''

\href{https://www.go-fair.org/fair-principles/}{Learn more about the
FAIR Principles}
\href{https://www.go-fair.org/wp-content/uploads/2022/01/FAIRPrinciples_overview.pdf}{FAIR
Principles Info PDF}
\end{block}
\end{frame}

\begin{frame}{Reproducible Science Tools}
\protect\hypertarget{reproducible-science-tools}{}
\begin{itemize}
\tightlist
\item
  R
\item
  Latex, Markdown
\item
  Python
\item
  Github
\item
  Docker, VMs
\item
  AWS CDK
\end{itemize}
\end{frame}

\begin{frame}{What is Github?}
\protect\hypertarget{what-is-github}{}
A platform to store and collaborate on code.

\href{https://desktop.github.com/}{Get Github Desktop}

\href{https://training.github.com/downloads/github-git-cheat-sheet/}{Get
Github Cheatsheets}
\end{frame}

\begin{frame}{Github Concepts:}
\protect\hypertarget{github-concepts}{}
\begin{itemize}
\tightlist
\item
  Repository: a logical unit of storage for maintaining code/other
  assets. Can contain many folders, files.
\item
  .gitignore: a file (with dot prefix) to specify what files should NOT
  be pushed to Github
\item
  .gitkeep: a file (with dot prefix) to specify that a folder should be
  kept if blank as part of the project structure. This is helpful when
  creating templates.
\item
  `commit': create a record/snapshot of all files (or specific files) at
  a moment in time. You may tag collaborators, add a title and notes.
\item
  `push': `storing' the result of prior commits
\item
  `pull': `extracting' the result of prior commits
\item
  `diff': difference (line-specific) between two commits. Helpful when
  trying to remember when and where code edits where made.
\end{itemize}
\end{frame}

\begin{frame}[fragile]{What is R?}
\protect\hypertarget{what-is-r}{}
Statistical programming language. Similar to \texttt{S}

\begin{block}{What is RStudio?}
\protect\hypertarget{what-is-rstudio}{}
Integrated development environment (IDE) for R
\end{block}

\begin{block}{What is RMarkdown?}
\protect\hypertarget{what-is-rmarkdown}{}
Reproducible notebook (akin to Jupyter Notebooks in Python) where you
can write a narrative using Markdown syntax and embed code/plots
throughout.
\end{block}
\end{frame}

\begin{frame}[fragile]{What are R packages?}
\protect\hypertarget{what-are-r-packages}{}
Think of CRAN as an equivalent to iOS App Store, or Google Play Store -
CRAN is where you download published R packages (directly from R,
RStudio).

A developer can also make packages available directly in source code, or
via tools like Github, for example:

\begin{verbatim}
install.packages("devtools")
devtools::install_github("nelsonroque/ruf")
\end{verbatim}
\end{frame}

\begin{frame}{R Markdown}
\protect\hypertarget{r-markdown}{}
This is an R Markdown presentation. Markdown is a simple formatting
syntax for authoring HTML, PDF, and MS Word documents. For more details
on using R Markdown see \url{http://rmarkdown.rstudio.com}.

When you click the \textbf{Knit} button a document will be generated
that includes both content as well as the output of any embedded R code
chunks within the document.
\end{frame}

\begin{frame}{Endnote}
\protect\hypertarget{endnote}{}
\begin{block}{What is Endnote?}
\protect\hypertarget{what-is-endnote}{}
A citation manager for research projects.

Here is a \href{https://guides.ucf.edu/citations-endnote}{guide from the
UCF library with installation links}.
\end{block}

\begin{block}{Why Endnote?}
\protect\hypertarget{why-endnote}{}
\begin{itemize}
\tightlist
\item
  no need for manual tracking (deleting, updating) of references
\item
  change citation format at click of a button
\item
  capture PDFs for any citations at click of a button
\end{itemize}
\end{block}

\begin{block}{Downloads:}
\protect\hypertarget{downloads}{}
\begin{itemize}
\tightlist
\item
  \href{http://ezproxy.library.ucf.edu/loggedin/EndNote20SiteInstaller.zip}{Mac}
\item
  \href{http://ezproxy.library.ucf.edu/loggedin/EndNote20.zip}{Windows}
\end{itemize}
\end{block}
\end{frame}

\begin{frame}{Interested in learning more R?}
\protect\hypertarget{interested-in-learning-more-r}{}
\href{https://github.com/nelsonroque/contextlab_intro_datascience}{There's
a Github repo for that!}
\end{frame}

\begin{frame}{Intro to Reproducible Science: A Summer Workshop}
\protect\hypertarget{intro-to-reproducible-science-a-summer-workshop}{}
The Context Lab at UCF
(\url{https://sciences.ucf.edu/psychology/contextlab/}) is gauging
interest in a Github + R workshop over Summer 2022.

Please fill out this form if interested in attending. If you know
someone that is interested in attending, please feel free to share this
form.

Scan below or \href{https://forms.office.com/r/8j6QarEHwa}{click Here to
fill out the Interest Form}
\end{frame}

\begin{frame}{Resources}
\protect\hypertarget{resources}{}
\begin{block}{Books}
\protect\hypertarget{books}{}
\begin{itemize}
\tightlist
\item
  \href{https://r4ds.had.co.nz/}{R for Data Science}
\item
  \href{https://adv-r.hadley.nz/index.html}{Advanced R}
\item
  \href{http://www.cookbook-r.com/}{R Graphics Cookbook}
\item
  \href{https://www.tidytextmining.com/}{Text Mining with R}
\item
  \href{https://nceas.github.io/sasap-training/materials/reproducible_research_in_r_fairbanks/}{Reproducible
  Analyses with R}
\end{itemize}
\end{block}
\end{frame}

\begin{frame}{Resources}
\protect\hypertarget{resources-1}{}
\begin{block}{Cheatsheets}
\protect\hypertarget{cheatsheets}{}
\begin{itemize}
\tightlist
\item
  \href{https://www.rstudio.com/resources/cheatsheets/}{List of
  Cheatsheets}
\item
  \href{https://raw.githubusercontent.com/rstudio/cheatsheets/main/data-visualization.pdf}{ggplot2}
\item
  \href{https://raw.githubusercontent.com/rstudio/cheatsheets/main/data-transformation.pdf}{dplyr}
\item
  \href{https://raw.githubusercontent.com/rstudio/cheatsheets/main/tidyr.pdf}{tidyr}
\item
  \href{https://raw.githubusercontent.com/rstudio/cheatsheets/main/data-import.pdf}{Data
  import with readr, readxl, and googlesheets4 cheatsheet}
\item
  \href{https://raw.githubusercontent.com/rstudio/cheatsheets/main/purrr.pdf}{Apply
  functions with purrr}
\item
  \href{https://raw.githubusercontent.com/rstudio/cheatsheets/main/strings.pdf}{String
  manipulation}
\item
  \href{https://raw.githubusercontent.com/rstudio/cheatsheets/main/lubridate.pdf}{Working
  with dates and times}
\item
  \href{https://raw.githubusercontent.com/rstudio/cheatsheets/main/rmarkdown.pdf}{RMarkdown}
\end{itemize}
\end{block}
\end{frame}

\begin{frame}{Resources}
\protect\hypertarget{resources-2}{}
\begin{block}{Visualization}
\protect\hypertarget{visualization}{}
\begin{itemize}
\tightlist
\item
  \href{https://r-charts.com/}{R-Charts}
\item
  \href{https://cran.r-project.org/web/packages/cowplot/vignettes/introduction.html}{Design
  your own Grid Plot}
\end{itemize}
\end{block}

\begin{block}{Stats}
\protect\hypertarget{stats}{}
\begin{itemize}
\tightlist
\item
  \href{https://www.cyclismo.org/tutorial/R/}{Intro to R Stats}
\item
  \href{https://quantdev.ssri.psu.edu/tutorials/r-bootcamp-introduction-multilevel-model-and-interactions}{Multilevel
  Modeling Primer 1}
\item
  \href{https://rpubs.com/rslbliss/r_mlm_ws}{Multilevel modeling Primer
  2}
\end{itemize}
\end{block}
\end{frame}

\begin{frame}{Resources}
\protect\hypertarget{resources-3}{}
\begin{block}{Blogs}
\protect\hypertarget{blogs}{}
\begin{itemize}
\tightlist
\item
  \href{https://blog.revolutionanalytics.com/}{Revolution Analytics}
\item
  \href{https://www.r-bloggers.com/}{RBloggers}
\end{itemize}
\end{block}

\begin{block}{Interactive Learning Tools}
\protect\hypertarget{interactive-learning-tools}{}
\begin{itemize}
\tightlist
\item
  \href{https://swirlstats.com/}{Swirl - Interactive R}
\item
  \href{https://www.datacamp.com/courses/free-introduction-to-r}{DataCamp}
\item
  \href{https://www.codecademy.com/catalog/language/r}{CodeAcademy}
\end{itemize}
\end{block}
\end{frame}

\begin{frame}{Thank you!}
\protect\hypertarget{thank-you}{}
Feel free to email me at \url{nelson.roque@ucf.edu} if you have any
questions.
\end{frame}

\end{document}
